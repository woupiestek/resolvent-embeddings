\documentclass[sort&compress]{elsarticle}
\usepackage{amssymb, amsmath, amsthm}
\usepackage[all]{xy}
\usepackage{cite}
\usepackage{url}

\theoremstyle{plain}
\newtheorem{theorem}{Theorem}
\newtheorem{lemma}[theorem]{Lemma}

\theoremstyle{definition}
\newtheorem{definition}[theorem]{Definition}
\newtheorem{example}[theorem]{Example}

\theoremstyle{remark}
\newtheorem{remark}[theorem]{Remark}

\newcommand\hide[1]{}
\newcommand\key[1]{\emph{#1}\label{#1}}
\newcommand\cat\mathcal
\newcommand\set[1]{\left\{#1\right\}}
\newcommand\exlex{_\mathrm{ex/lex}}\newcommand\exreg{_\mathrm{ex/reg}}
\newcommand\of:

\begin{document}
\begin{frontmatter}
\title{Resolvent embeddings}

\author[W. P. Stekelenburg]{Wouter Pieter Stekelenburg}
\address{Faculty of Mathematics, Informatics and Mechanics,
University of Warsaw,
Banacha 2,
02-097 Warszawa,
Poland}
\ead{w.p.stekelenburg@gmail.com}
\fntext[W. P. Stekelenburg]{Corresponding author; tel.: +31624543216}

\begin{abstract}
\end{abstract}

\begin{keyword}
\end{keyword}

%AMS 18A22

\end{frontmatter}


\section{Resolvent Embeddings}
As the title suggests this paper is about the following kind of functor.

\begin{definition} Let $F\of \cat C\to\cat D$ be an arbitrary functor. For each object $X$ of $\cat D$, a \key{resolution} is an object $Y$ in $\cat C$ together with a morphism $r:FY\to X$ with the following two properties.
\begin{enumerate} 
\item The morphism $r$ is a regular epimorphism.
\item For each morphism $f\of FZ\to X$ there is a $g\of Z\to Y$ such that $f=r\circ Fg$.
\[ \xymatrix{
& FY \ar[d]^{r} \\
FZ\ar[r]_f \ar@{.>}[ur]^{Fg} & X
}\]
\end{enumerate}
A functor $F:\cat C\to\cat D$ is \key{resolvent} if each object of $\cat D$ has a resolution. A \emph{resolvent embedding} is a resolvent functor that is also fully faithful.
\end{definition}

Resolvent embeddings are particularly interesting when they are also \emph{ex/reg completions}.

\begin{definition} A regular functor $F\of\cat C\to\cat D$ from a regular category $\cat C$ to an exact category $\cat D$ is an \key{ex/reg completion} of $\cat C$ if every regular functor $G\of \cat C\to\cat E$ to an exact category $\cat E$ factors through $F$ up to isomorphism in an up to isomorphism unique way.

A finite limit preserving functor $F\of\cat C\to\cat D$ from a category with finite limits $\cat C$ to an exact category $\cat D$ is an \key{ex/lex completion} of $\cat C$ if every finite limit preserving functor $G\of \cat C\to\cat E$ to an exact category factors through $F$ up to isomorphism in an up to isomorphism unique way.
\end{definition}\hide{It is clear that completions are fully faithful? How does this follow from the universal property?}

\begin{theorem} Let $F:\cat C\to\cat C\exreg$ be an ex/reg completion of a regular category $\cat C$, let $G\of \cat C \to \cat C\exlex$ be the ex/lex completion and let $H$ be the up to isomorphism unique functor $H\of\cat C\exreg \to \cat C\exlex$ that satisfies $HG\simeq F$. 
The following are equivalent.
\begin{enumerate}
\item $F$ is resolvent.
\item $H$ has a fully faithful right adjoint.
\item For each exact $\cat D$, each functor $K\of \cat C \to \cat D$ that preserves finite limits has a left Kan extension along $F$ that also preserves finite limits.
\end{enumerate}
\end{theorem}

\begin{proof} Lemma \ref{resolvent to left} show that 1 implies 3 and lemma \ref{left to resolvent} shows that 3 implies 1 because ex/reg completions are fully faithful and preserve finite limits.



The ex/lex completion is resolvent because %do we need a lemma here?


\hide{Point ahead to lemmas down this paper

ThA -> LKE,converse,adjunct
adjunct -> LKE


ThB -> lccc, closubclass, ...


}
\end{proof}

Resolvent embeddings also play a role in topos theory.

\begin{theorem} Let $F:\cat C\to\cat C\exlex$ be the ex/reg completion of a regular category $\cat C$, then the following are equivalent.
\begin{enumerate}
\item $\cat C$ has \emph{weak dependent products} and a \emph{generic monomorphism}. \hide{these need to be defined somewhere}
\item $\cat C\exreg$ is a topos and $F$ is resolvent. 
\end{enumerate}
\end{theorem}

\begin{proof}
\hide{Point ahead to lemmas down this paper}
\end{proof}

\hide{conclusion... }

\section{Kan extensions}
This section uses \emph{pseudo-equivalence relations} in order to construct left Kan extensions along certain resolvent embeddings.

\newcommand\ri{^*}
\newcommand\id{\mathrm{id}}
\newcommand\dom{\mathrm{dom}}
\newcommand\cod{\mathrm{cod}}
\begin{definition} In any category with finite limits $\cat C$, a parallel pair $f,g\of X\to Y$ is a \key{pseudo-equivalence relation}, if there are morphisms $r:Y\to X$, $s:X\to X$ and $t:X\times_YX \to X$--where $X\times_YX = \set{(x,y)\of X\times X|f(x)=g(y)}$ is the pullback of $f$ and $g$--such that the following equations hold.
\begin{align*}
f\circ r &=\id_Y & g\circ r &= \id_Y\\
f\circ s &= g & g\circ s &= f\circ s\\
f\circ t &= f\circ g\ri(f) & g\circ t &= g\circ f\ri(g)
\end{align*}

Here $f\ri(g)$ stand for the pullback of $g$ along $f$ and $\id_Y$ for the identity morphism of $Y$.

A \emph{morphism of pseudo-equivalence relations} $(f,g) \to (h,k)$ is a pair of morphisms $l_1\of \dom(f) \to \dom(h)$ and $l_0\of \cod(f) \to \cod(h)$ such that the following equations hold.
\begin{align*}
l_0\circ f &= h\circ l_1 & l_0\circ g &= k\circ l_1\\
\end{align*}

Here $\dom(f)$ is the domain of $f$ and $\cod(f)$ is the codomain of $f$.

Two such morphisms $l=(l_0,l_1),m=(m_0,m_1):(f,g) \to (h,k)$ are \emph{equivalent} is there is a morphism $n\of \cod f \to \dom h$ such that $h\circ n = l_0$ and $k\circ n = l_1$. Such an $n$ is an \emph{equivalence} of $l$ and $m$.
\end{definition}

\begin{remark} For a category with finite limits being exact means that every pseudoequivalence relation has a coequalizer, and that these coequalizers are stable under pullback.
\end{remark}


\newcommand\di{_!}
\begin{lemma} Let $F\of\cat C\to\cat D$ be a finite limit preserving resolvent embedding between categories with finite limits. For each exact $\cat E$ each finite limit preserving functor $G\of\cat C\to\cat E$ has a finite limit preserving left Kan extension $F\di(G)$ along $F$. Moreover, $F\di(G)F$ is naturally isomorphic to $G$.\label{resolvent to left}\end{lemma}

\newcommand\pser\mathbf
\begin{proof}
For each object $D$ of $\cat D$ there is a pseudo-equivalence relation $\pser D$ in $\cat C$ such that a resolution of $D$ is the coequalizer of $F(\pser D)$by lemma \ref{induced psers}.
The functor $G$ sends $\pser D$ to a pseudo-equivalence relation $G(\pser D)$ in $\cat D$.
Let $F\di(G)(D)$ be the coequalizer of $G(\pser D)$, which is the image of $\pser D$ after application of $G$.

Each morphism $d\of D\to D'$ of $\cat D$ induces a morphism of pseudoequivalence relations $\pser{d}:\pser{D}\to\pser{D'}$ by lemma \ref{induced psers}.
The functor $G$ sends $\pser{d}$ to a morphism $G(\pser{d})\of G(\pser D)\to G(\pser{D'})$.
Let $F\di(G)(d)$ be the morphism that $G(\pser{d})$ induces between $F\di(G)(D)$ and $F\di(G)(D')$.

This mapping $F\di(G)$ is a finite lifting preserving functor for the following reasons. Because $F$ is resolvent, all the equations of morphisms in $\cat D$ that $F\di(G)$ must preserve are backed by equivalences of morphisms between pseudoequivalence relations in $\cat C$. By lemma \ref{equivalence implies equality} equivalence between morphisms of pseudoequivalence relation induce equalities between the coequalizers that form the image of $F\di(G)$.%more?

For each $H\of\cat D\to\cat E$ and each natural transformation $\eta\of G \to HF$ there is a unique $\eta'\of F\di(G)\to H$ such that $\eta'_{FX}=\eta_X$. It is the pointwise factorization of morphisms $GX_0\to HX$ through coequalizers of pseudo-equivalence relations.

For $D = FX$, $\id_{FX}$ is a resolution. It induces the trivial pseudoequivalence relation $\pser D = (\id_X,\id_X)$, and $\id_{GX}$ is a coequalizer of $GX$. Since the properties that must satisfy $F\di(G)$ make it unique up to isomorphism, $F\di(G)F$ must be naturally isomorphic to $G$. 
\end{proof}

\begin{lemma} Let $F\of\cat C\to\cat D$ be a finite limit preserving resolvent embedding between categories with finite limits.
\begin{enumerate}
\item For each $D$ of $\cat D$ there is a pseudoequivalence relation $\pser D$ in $\cat C$ such that the resolution of $d$ is the coequalizer of $F(\pser D)$. 
\item For each morphism $d\of D\to D'$ there is a morphism $\pser d\of\pser D\to \pser{D'}$ that commutes with the resolutions of $D$ and $D'$ and $F(\pser d)$.
\end{enumerate}\label{induced psers}
\end{lemma}

\begin{proof} 
\begin{enumerate}
\item Let $d_0: FD_0\to D$ be a resolution of $D$, and let $d_1:FD_1 \to FD_0\times_DFD_0$ be a resolution of the domain of the pullback $d_0\ri(d_0)$. Since $F$ is an embedding, the morphisms $\pi_i\circ d_1\of FD_1\to FD_0$--where $\pi_i$ for $i=0,1$ are the projections of the pullback--are the image of a parallel pair of morphism $f,g\of D_1 \to D_0$ in $\cat C$.

An $r:D_0\to D_1$ that satisfies $f\circ r = g\circ r = \id_{D_0}$ comes from the diagonal $FD_0 \to FD_0\times_DFD_0$ because $F$ is resolvent. An $s\of D_1\to D_1$ that satisfies $f\circ s = g$ and $g\circ s = f$ comes from the swapping map $(x,y)\mapsto(y,x)\of FD_0\times_DFD_0\to FD_0\times_DFD_0$ for the same reason. Finally, a $t\of D_1\times_{D_0} D_1\to D_1$ that satisfies $f\circ t = f\circ g\ri(f)$ and $g\circ t = g\circ f\ri(g)$ comes from the map $(x,y) \mapsto (Ff(x),Fg(y)) \of FD_1\times_{FD_0} FD_1\to FD_0\times_DFD_0$.

\item Let $d \of D\to D'$ be a morphism of $\cat D$ and let $a\of FD_0\to D$ and $b\of FD'_0\to D'$ be resolutions. There is a $d_0\of D_0\to D'_0$ such that $b\circ F(d_0) = d\circ a$ because $F$ is resolvent. In turn, $(Fd_0,Fd_0)\of FD_0\times_DFD_0\to FD'_0\times_DFD'_0$ induces a map $d_1\to D_1\to D'_1$ that commutes with resolutions $a'\of FD_1\to FD_0\times_DFD_0$ and $b'\of FD'_1\to FD'_0\times_DFD'_0$. Because $F$ is fully faithful, $\pser d = (d_0,d_1)$ is a morphism of pseudoequivalence relations such that $F(\pser d)$ commutes with the resolutions and with $d$.
\end{enumerate}
\end{proof}

\begin{lemma} Let $f,g\of(a,b)\to(c,d)$ be equivalent morphisms of pseudo-e\-qui\-va\-len\-ce relations in an exact category. Both induce the same morphism between the coequalizers. \label{equivalence implies equality} \end{lemma}

\begin{proof} An equivalence $h$ between $f$ and $g$ factors the pair $(f_0,g_0)$ through $c$ and $d$. Therefore the coequalizer $e$ of $c$ and $d$ must satisfy $e\circ f=e\circ g$. This is sufficient, since the induced morphisms are the factorizations of $e \circ f$ and $e\circ g$ though the coequalizer of $a$ and $b$.
\end{proof}

Lemma \ref{resolvent to left} has a kind of converse.

\begin{lemma} Let $F\of\cat C\to\cat D$ be a finite limit preserving fully faithful between categories with finite limits and let $\cat D$ be exact. If for each exact $\cat E$ each finite limit preserving functor $G\of\cat C\to\cat E$ has a finite limit preserving left Kan extension $F\di(G)$ along $F$ then $F$ is resolvent. \label{left to resolvent}\end{lemma}

\begin{proof} Define $\cat E$ as follows. Objects are regular epimorphisms $FX\to Y$ in $\cat D$. A morphism $e\to e'$ is a morphism $f\of\cod(e)\to\cod(e')$
such that $f\circ e$ factors through $e'$. This category $\cat E$ is easily proved exact. Let $G\of \cat C\to\cat E$ map $X$ to $\id_{FX}$. This functor preserves finite limits. Any left Kan extension $F\di(G)$ sends each object $D$ of $\cat D$ to a resolvent embedding. Therefore $F$ is resolvent.
\end{proof}

\hide{ The intuition is equational and essentially algebraic logic. Finite limit preserving functors preserve these.}

\end{document}

